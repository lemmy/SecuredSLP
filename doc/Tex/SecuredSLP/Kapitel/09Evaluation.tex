\section{Evaluation}
The integration and implementation of suggested solutions was successful. The specifications of common SLP were extended for confidentiality, integrity, authentication and replay attack prevention. The result of this extension is the SecuredSLP specification and it provides fundamental security features which are needed for a secure communication in an open network. Current SecuredSLP is implemented as a ``proof of concept'' version \textcolor{red}{kann man das so sagen?} and works with TGDH as its group key agreement protocol. But the SecuredSLP specification is not bound to a special protocol, any other group key agreement protocol could be integrated as well as TGDH. Also it is possible to add or change the cipher-suite and use other cryptographic algorithms. In current state the implementation is limited to RSA.\\\\
But there aren't just positive aspects about the extension to SecuredSLP. Several problems arrive with the usage of different group key agreement protocols. First of all the secured groups demand extra discovery iteration. In the first discovery iteration a user is just able to discover all plaint scopes. To discover the security scopes/groups the user needs second discovery iteration. That means that the time to discover a security group is doubled and this is process deterioration especially in an open network with low performance devices. Another problem is the group key agreement protocol with forward/backward security features. The rekeying phase of the group key agreement protocol always disturbs the workflow of the service location protocol. A short time after rekeying the old keys are useless and won't be accepted for the communication. So if the network is unstable and dynamic the rekeying phase becomes a thread. Many users join/leave the security group so the security group will stuck in the rekeying phase (compare section \ref{sec:DoS}). For that purpose we got validation intervals for the group shared keys. That means we dynamically keep several group shared keys. The number of the keys, which are valid at the time, depends on the network behavior and changes on demand. We keep that much keys as necessary and that little as possible. \textcolor{red}{``so viel wie n�tig, so wenig wie m�glich'' (gibts daf�r eine bessere �bersetzung?)} In that way we avoid the intervals where the security group members aren't able to communicate and we still keep the security of the group shared key(s) high. This feature is already implemented in the current SecuredSLP version but not tested. The right balance for that algorithm isn't set yet. Although the scale ability (\textcolor{red}{skalierbarkeit auf englisch?}) of SecuredSLP with a big amount of users in a security group wasn't test in praxis yet. But we assume that it should work just fine with several hundred users (\textcolor{red}{hab ich mal so als Vermutung aufgestellt}).  