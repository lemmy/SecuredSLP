\section{Evaluation}\label{sec:evaluation}
The integration and implementation of the suggested solutions were successful.
The specifications of common SLP were extended for confidentiality, integrity,
authentication and replay attack prevention. The result of this extension is
the SecuredSLP specification and it provides fundamental security features
which are needed for a secure communication in an open network. Current
SecuredSLP is implemented as a ``proof of concept'' version and works with TGDH
as its group key agreement protocol. But the SecuredSLP specification is not
bound to a special protocol, any other group key agreement protocol could be
integrated instead. It is also possible to add or change the cipher-suite and
use other cryptographic algorithms. In current state the implementation is
limited to RSA.\\\\ But there are not only positive aspects resulting from the
extension to SecuredSLP. Several problems occur by useging different
group key agreement protocols. First of all the secured groups demand extra
discovery iteration. In the first discovery iteration a user is only able to
discover all plain scopes. To discover the security scopes/groups the user
needs a second discovery iteration. That means that the time to discover a
security group is doubled and this is process deterioration especially in an
open network with low performance devices. Another problem is the group key
agreement protocol with forward/backward secrecy features. The rekeying phase of
the group key agreement protocol always disturbs the workflow of the service
location protocol. A short time after rekeying old keys are useless and won't
be accepted for the communication. So if the network is unstable and dynamic
the rekeying phase becomes a thread. If many users join/leave the security
group the security group will be stuck in the rekeying phase (compare section
\ref{sec:DoS}). For that purpose we get validation intervals for the group
shared keys. That means we dynamically keep several old group shared keys
valid. The number of the keys, which are valid at a time, depends on the
network behavior and changes on demand. We reduce the number of keys valid at a
time. In that way we minimize the intervals where the security group members
are not able to communicate but we still keep the security of the group shared
key(s) high. This feature is already implemented in the current SecuredSLP
version but not tested. The right balance for that algorithm is not set yet.
Also the scale ability of SecuredSLP with a large number of users in a security
group has not been tested practically yet. But we assume that it will work
with several hundred users. We did not apply any trust model to the SecuredSLP
implementation. That means anyone, who wants to join a security group, can get
the access.\\ \hrule The extension from common SLP to SecuredSLP is a first
step to increase the security of the service location protocol especially for
the use in a new environment like an open network. SLP was not originally
designed for the usage in open networks. In the first and second sections of
this paper we introduce the workflow of a service discovery protocol itself and
its issues and the problems while using it in an open network. In section
\ref{sec:slp-overview} some useful solutions are presented and discussed how to
raise the security of this protocol. An overview of the suggested solutions can
be found in section \ref{sec:conclusion}. In sections 5 - 9 the introduced
solutions were applied to the SLP specification and a ``proof of concept''
version of SecuredSLP was implemented. Those sections describe how and why the
integration of security has been done.
