\section{Conclusion and Future Work}
With SecuredSLP we tried to adopt a service location protocol to a new environment, to open networks. Compared to SLP the ``proof of concept'' implementation of SecuredSLP already delivers many new changes and advantages. The common security features like authentication, integrity and confidentiality are now a part of the extended protocol. Anyway the current implementation is just a ``proof of concept'' and isn't complete yet.\\
In the next version it would be useful to apply more cryptographic algorithms to provide more flexibility for the users and offer several degrees of security. Although it is necessary to apply a trust model to the protocol. The best cryptographic algorithms are worthless if the communication partners can't trust each other. There are already good solutions, like PKI, WoT etc., to establish trust in a network. Those could be also used here but it's important to decide with trust model should be used and how should it be integrated into SecuredSLP.\\  
Other important work for the future is to proof how the implemented version of SecuredSLP works in praxis especially in large networks with more than hundred peers. The SecuredSLP provides the facility to configure the protocol behavior like keeping a specified amount of old group shared keys valid. But those features weren't tested yet. It is important to know how the extended protocol scales in praxis to decide in which areas and with what amount of users it could be used.