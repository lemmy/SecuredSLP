\section{Integration: group key agreement protocol with SLP}
There are two main approaches how to integrate a group key agreement protocol into the SLP. It is possible to integrate the group key agreement hardcoded or as a module adjacent. Both ways have benefits and disadvantages. 

\subsection{Hardcoded integration}
With this solution we have to take the SLP specification apart and change many methods to combine a group key agreement protocol with SLP. Such integration means that we do not only need some changes but a completely new communication workflow with new message formats and timeout behavior. Furthermore the group key agreement protocol could not be replaced anymore. That means all users would be bound to a special group key agreement protocol. That could be a clean solution but it would require a strong change in the SLP and the implementation would become much more complicated. Furthermore the SLP would not be compatible anymore to the versions below. This approach would deliver a new protocol which probably would scale better compared to the modular integration solution.

\subsection{Modular integration}
With this solution we do not need many changes in the SLP specification. We use a group key agreement protocol as a module which we combine with SLP without fully integrating it into the SLP specification. That way the module can easily be replaced anytime and we are not bound to a special group key agreement protocol. It is also possible to use many several group agreement protocols at the same time. Our intention was to upgrade the SLP with minimum changes but to solve the requested requirements as discussed in section \ref{sec:conclusion}. Furthermore this solution can be applied much easier on current SLP implementation than on a hardcoded integration.\\
For this purpose we did minor changes in the source code. As discussed above we took TGDH as our group agreement protocol. To make SLP work with TGDH we had to change the header of SLP messages and implement additional functions into SLP to communicate with our group agreement protocol. A disadvantage of this method is extra communication overhead because SLP and the group key agreement protocol do not work synchronized together.
