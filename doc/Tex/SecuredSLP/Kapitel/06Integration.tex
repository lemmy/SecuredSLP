\section{Integration of a group key agreement protocol into SLP}
There are two main approaches how to integrate a group key agreement protocol into the SLP. It is possible to integrate the group key agreement hardcoded into SLP or as a module adjacent to SLP. Both got their benefits and disadvantages. 

\subsection{Hardcoded integration}
With this solution we have to take the SLP source code apart and change many methods to combine a group key agreement protocol with SLP. That could be a neatly solution but would change the SLP too much and the implementation would be much complicated and expensive in time. The SLP probably wouldn't be compatible to the versions below and that isn't our goal. This approach would deliver a new protocol which probably would scale better as the modular integration solution. \textcolor{red}{eventuell paar Einzelheiten erg�nzen}

\subsection{Modular integration}
With this solution we don't make many changes in the SLP source code. We use a group key agreement protocol as a module which we combine with SLP without fully integrates it into the SLP source code. In this way the module can easily be replaced anytime. So we are not bound to a special group agreement protocol. Also it is possible to use many several group agreement protocols at the same time if needed.\\
For this purpose we have minor changes in the source code. Like discussed above we took TGDH as our group agreement protocol. To make SLP work with TGDH we had to change the header of SLP messages and implement additional functions into SLP to communicate with our group agreement protocol. \textcolor{red}{hier vielleicht mehr details}

\subsubsection{Workflow of SLP with a modular integration}
\textcolor{red}{hier noch ein Bild einf�gen}